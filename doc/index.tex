% nurdlib, NUstar ReaDout LIBrary
%
% Copyright (C) 2015-2017, 2020-2022, 2024
% Bastian Löher
% Hans Toshihide Törnqvist
%
% This library is free software; you can redistribute it and/or
% modify it under the terms of the GNU Lesser General Public
% License as published by the Free Software Foundation; either
% version 2.1 of the License, or (at your option) any later version.
%
% This library is distributed in the hope that it will be useful,
% but WITHOUT ANY WARRANTY; without even the implied warranty of
% MERCHANTABILITY or FITNESS FOR A PARTICULAR PURPOSE.  See the GNU
% Lesser General Public License for more details.
%
% You should have received a copy of the GNU Lesser General Public
% License along with this library; if not, write to the Free Software
% Foundation, Inc., 51 Franklin Street, Fifth Floor, Boston,
% MA  02110-1301  USA

\documentclass{article}

\usepackage{fancyvrb}

\author{Hans T\"ornqvist (h.toernqvist@gsi.de)}
\title{nurdlib - the NUstar ReaDout LIBrary}

\begin{document}

\maketitle




\begin{Verbatim}[frame=single,numbers=left]
git clone http://???/nurdlib.git
make -C nurdlib
make -C nurdlib fuser_drasi
echo 'CRATE("test") {}' > nurdlib.cfg
./nurdlib/build_*/m_read_meb.drasi ...
\end{Verbatim}

The above snippet is a loose minimal example of running a DAQ using
\emph{nurdlib} and \emph{drasi}~\cite{bib:drasi}.

This document will first introduce the basics, then cover common use cases and
questions, and at last delve deeper into the inner workings of this library.



\section{Introduction}

If you are in dire need to read out some commonly used data acquiring hardware
for nuclear physics or similar, there is a chance that this piece of software
could help.

Rather than being another repository of code for copy-n-pasting, nurdlib was
conceptualised to carefully configure, initialize, read, and verify a set of
modules in run-time. Due to the wide range of modules and use-cases in this
field it will probably not cover all bases, but it will most probably save
time and work in the pursuit of good data integrity.

From a user's point of view, there are two big interfaces: the glue to a DAQ
backend, and the run-time configuration.

Let's first look at a minimal pseudo code example of glue code between nurdlib
and a single-event non-buffering DAQ backend. Comments are left out for visual
clarity, but explanations follow below.

\begin{Verbatim}[frame=single,numbers=left]
#include <nurdlib.h>

struct Crate *g_crate;
struct CrateTag *g_tag;

void
backend_setup(void)
{
  g_crate = nurdlib_setup(NULL, "nurdlib.cfg", NULL, NULL);
  g_tag = crate_get_tag_by_name(g_crate, NULL);
}

void
backend_shutdown(void)
{
  nurdlib_shutdown(&g_crate);
}

void
backend_readout(void *a_buf, size_t a_buf_bytes)
{
  struct EventBuffer eb;
  uint32_t ret;

  eb.ptr = a_buf;
  eb.bytes = a_buf_bytes;

  crate_tag_counter_increase(g_crate, g_tag, 1);

  ret = crate_readout_dt(g_crate);
  if (0 != ret) log("crate_readout_dt failed with 0x%x.", ret);

  ret = crate_readout(g_crate, &event_buffer);
  if (0 != ret) log("crate_readout failed with 0x%x.", ret);

  crate_readout_finalize(g_crate);
}
\end{Verbatim}

\begin{description}
\item[1] nurdlib master include file for the most common functionality.
\item[3] The crate type contains the state of the read-out above the level of
modules. Typically, only a single crate is necessary per process.
\item[4] Module tagging is used to group modules into different read-out
conditions, in order to accurately keep track of module counters in complex
configurations. For a simple configuration, the automatic tag named "Default"
will suffice.
\item[6--11] This represents the setup function called by the DAQ backend.
\item[9] nurdlib does nothing before this function is called, except for
initialisation of internal global variables. The first argument may be a
logging callback with a specific signature, but a null value will default to
standard stdio calls to stdout and stderr. The second argument is the filename
from which to read the high-level run-time configuration.
\item[10] A null value will grab the module tag named "Default" from the crate
created on the previous line.
\item[13--17] This represents the shutdown function called by the DAQ backend,
if available.
\item[16] This cleans up all resources taken by the crate. This call is mostly
optional, depending on the DAQ backend and automatic resource cleanup by the
system.
\item[19--37] This represents the readout function called by the DAQ backend.
The readout condition depends on the backend, and could fire on any trigger,
or some groups of triggers etc.
\item[22] An event-buffer is a description of the memory provided by the DAQ
backend for event data readout.
\item[23] The return value of several nurdlib readout functions.
\item[25--26] Sets up the event-buffer.
\item[28] Increments the event-counter for all modules with the specified tag.
A tag event-counter can eventually be linked with a scaler channel to support
multi-event readout.
\item[30] At this point, nurdlib expects the DAQ to be in dead-time. This
function will read the minimal set of data from all modules in order to verify
synchronisation, for example event counters and event data sizes. Eventually,
a dead-time release callback can be provided, which nurdlib will call at the
end of this function if the user configuration has enabled this feature.
\item[33] Performs readout of the event payloads of all modules that reported
having data during dead-time.
\item[36] Finalises the readout of the full crate by checking and updating the
state to be ready for the next readout. Most importantly, a failed readout may
require immediate reinitialisation of hardware in order to lift hard busy
conditions before a new readout is allowed.
\end{description}

The high-level configuration, which is read from ASCII-based configuration
files during library setup, may look like:

\begin{Verbatim}[frame=single,numbers=left]
CRATE("MyCrate") {
  CAEN_V775(0x00400000) {}
}
\end{Verbatim}

\begin{description}
\item[1] Crate name for pretty print-outs.
\item[2] This will instantiate the state required for a CAEN V775 TDC.
\end{description}

And that's all for a simple single event Caen TDC crate setup! Default
configurations can be found in the default configuration directory
(\emph{nurdlib/cfg/default/}), which are instantiated for each configured
module and user configurations are patched on top.

Would you like to have multi-event readout with a \emph{TRIDI} module with the
\emph{TRLO II} firmware, a configured time-range for a Mesytec QDC with
adaptive conversion time between an accepted event and readout, more verbose
logging, and disable the online configuration server? That should not be too
difficult:

\begin{Verbatim}[frame=single,numbers=left]
# Available levels (subsequent levels include previous logs):
#  info    - the most basic logs.
#  verbose - a lot more information for run-time debugging.
#  debug   - even more information for nurdlib debugging.
#  spam    - use with care! Prints during readout for the
#            and may fill up log-files very fast.
log_level = verbose
# The online server is a rather light path for changing the
# configuration while the DAQ is running, during dead-time
# after readout. It should have negligible impact on
# performance, but famous last words and all.
control_port = 0
CRATE("MyTrickyCrate") {
  # ACVT will poll until complete events have been read out
  # or timeout occurs, the exact behaviour depends on module
  # support and implementation. If the user code has supplied
  # a CVT-set callback, a floating CVT value will be adjusted
  # in run-time to reduce polling in the hope to reduce
  # dead-time.
  acvt = true

  # Tags are used to group modules for accurate event counting
  # in complex module set configurations. Generally, the value
  # of a scaler channel can be stored for a given set of tags,
  # and all modules touched by the set will be checked against
  # this value. In this specific case, we want to compare the
  # TRIDI master start scaler to the MQDC32 event counter.
  # There is always a "Default" tag that is reused here to leave
  # the glue code unchanged.
  TAGS("Default") {
    scaler_name = "tridi_master_start"
  }

  GSI_TRIDI(0x02000000) {
    # This lets nurdlib set the TRLO II multi-event limit in
    # this module based on the event buffers of all modules
    # tagged "Default".
    multi_event = "Default"

    # TRLO II master start scaler channel referenced in TAGS
    # above. This feature is only available in modules which
    # implement this feature.
    SCALER("tridi_master_start") {
      type = master_start
    }
  }

  MESYTEC_MQDC32(0x00400000) {
  }
}
\end{Verbatim}



\section{Common features and issues}

This section will cover the most common features and issues that arise when
building and using nurdlib.

\subsection{Make targets}

\begin{Verbatim}[frame=single,numbers=left]
make help
\end{Verbatim}
Prints and explains available targets.

\subsection{Environment variables}

Many environment variables control the build process, but a few also control
the run-time behaviour.

\subsubsection{Run-time variables}

\begin{description}
\item[NURDLIB\_DEF\_PATH] Points to the directory where default configuration
files reside. This variable will override the default
\emph{../nurdlib/cfg/default/} path.
\item[NTEST\_COLOR] The tests output by ntest prints ANSI color escape codes
by default. Setting NTEST\_COLOR=0 exactly will disable these codes.
\end{description}

The configuration files can also use environment variables when including
files:

\begin{Verbatim}[frame=single,numbers=left]
CRATE("MyCrate") {
  include "$CFG_PATH/my_config.cfg"
}
\end{Verbatim}

\subsubsection{Build variables}

The typical compilation and linking flags are:
\begin{description}
\item[CPPFLAGS] Pre-compiler flags, such as -I and -D. These are prepended to
the nurdlib pre-compiler flags to override search paths, e.g.
\emph{-Iuser\_header\_path}.
\item[CFLAGS] Compilation flags, such as -W*. These are appended to the
nurdlib compilation flags to override compilation switches, e.g.
\emph{-Wno-crazy-warning}.
\item[LDFLAGS] Linking flags, such as -L. These are prepended to the nurdlib
pre-compiler flags to override search paths, e.g.
\emph{-Luser\_library\_path}.
\item[LIBS] Libraries, such as -l and direct paths to static and dynamic
libraries. These are appended to the nurdlib libraries to cover for missing
libraries, e.g. \emph{-luser\_library} or directly
\emph{/path/extlib.so.1.2.3}.
\end{description}

Build modes, set with \emph{BUILD\_MODE}, and implications:
\begin{description}
\item[cov] Coverage compiled and linked in. This will insert instrumentation
into the code to track code paths that execute, and the final binaries will
run much slower.
\item[gprof] gprof support for profiling.
\item[pic] Position independent code. This is necessary to build the Python
online configuration support.
\item[release] Optimized build.
\end{description}

Several programs can be overridden:
\begin{description}
\item[AR] Static linker.
\item[BISON] Bison-like parser generator.
\item[CCACHE] Ccache for fast recompilations.
\item[CPPCHECK] Static analysis tool for C code.
\item[FLEX] Lexical analyser generator for Bison input.
\item[LATEX] Latex to dvi converter.
\item[PYTHON] Python support for live configuration.
\item[SED] GNU sed.
\end{description}

Drasi related flags:
\begin{description}
\item[DRASI\_PATH] Path to drasi, where \$DRASI\_PATH/bin/drasi-config.sh is
expected to exist.
\end{description}

TRLO II related flags:
\begin{description}
\item[TRLOII\_PATH] Path to trloii directory.
\item[{TRIDI,VULOM4,RFX1}\_FW] By default, nurdlib chooses the TRLO II
firmware versions by looking at the directories
\emph{\$TRLOII\_PATH/trloctrl/fw\_*}. This decision can be overridden with
these env-vars.
\end{description}



\section{A deeper dive}



\subsection{Tags}

Tags allow modules to be grouped together in several ways while still
supporting correct event counting. The "Default" tag is always created first.
For example:
\begin{Verbatim}[frame=single,numbers=left]
CRATE("Tags") {
  CAEN_V775(0x00010000) {} # Let's call this module A.
  TAGS("1")
  CAEN_V775(0x00020000) {} # Module B.
  TAGS("2")
  CAEN_V775(0x00030000) {} # Module C.
  TAGS("1", "2")
  CAEN_V775(0x00040000) {} # Module D.
\end{Verbatim}
This will put tag "Default" on module A, tag "1" on modules B and D, and tag
"2" on C and D. Every combination of tags creates a unique event counter, and
the progression of this counter must exactly match that of the event counters
of all modules with the same combination of tags. Table \ref{tab_tags} shows
an example of counter progression for the above configuration, where the tag
name relates to a trigger number.

\begin{table}
\begin{center}
\begin{tabular}{c|ccccc}
Trigger & Counter & Default & 1 & 2 & 1,2 \\
\hline
      - &         &       0 & 0 & 0 &   0 \\
      1 &         &       0 & 1 & 0 &   1 \\
      1 &         &       0 & 2 & 0 &   2 \\
      1 &         &       0 & 3 & 0 &   3 \\
      2 &         &       0 & 3 & 1 &   4 \\
      1 &         &       0 & 4 & 1 &   5 \\
      2 &         &       0 & 4 & 2 &   6 \\
      2 &         &       0 & 4 & 3 &   7 \\
\end{tabular}
\end{center}
\caption{Example of tags and split event-counting. Internally nurdlib does not
store a counter per tag, but per used combination of tags. This way some
modules can fire only on single triggers while other modules fire on all
triggers.}
\label{tab_tags}
\end{table}

Tags can be fetched for advancing the event counters:
\begin{Verbatim}[frame=single,numbers=left]
  /* The NULL tag means "Default". */
  g_tag[0] = crate_get_tag_by_name(g_crate, NULL);
  g_tag[1] = crate_get_tag_by_name(g_crate, "1");
  g_tag[2] = crate_get_tag_by_name(g_crate, "Timestamp");

  crate_tag_counter_increase(g_crate, g_tag[0], 1);
  if (1 == trigger)
    crate_tag_counter_increase(g_crate, g_tag[1], 1);
  if (10 == trigger)
    crate_tag_counter_increase(g_crate, g_tag[2], 1);

  crate_readout_dt(g_crate);
  crate_readout(g_crate, ...);
  crate_readout_finalize(g_crate);
}
\end{Verbatim}

It is also possible to link a scaler channel to a tag counter, in which case
the tag does not need to be increased by the user code:
\begin{Verbatim}[frame=single,numbers=left]
CRATE("Tags") {
  TAGS("1") {
    scaler_name = "my_scaler"
  }
  CAEN_V775(0x00020000) {}
  TAGS("2")
  CAEN_V830(0x00030000) {
    SCALER("my_scaler") {
      channel = 0
    }
  }
\end{Verbatim}
The first tag-combo will listen to the first channel of the V830 scaler module
to count the number of triggers sent to the V775.

Exactly how a module type defines a scaler channel depends on its
implementation. Below follows the available options as of writing.
\begin{Verbatim}[frame=single,numbers=left]
CAEN_V8{2,3}0:
  SCALER("my_scaler") { channel = 0..31 }
GSI_{TRIDI,VULOM4,RFX1}:
  SCALER("my_scaler") { type = accept_pulse                 }
  SCALER("my_scaler") { type = accept_trig  channel = 0..31 }
  SCALER("my_scaler") { type = ecl          channel = 0..31 }
  SCALER("my_scaler") { type = master_start                 }
  SCALER("my_scaler") { type = nim          channel = 0..31 }
\end{Verbatim}




\section{Building}

The nurdlib makefile is compatible with rather old versions of GNU make.
\begin{Verbatim}[frame=single,numbers=left]
make
\end{Verbatim}
This will produce a debug version of the static library \emph{libnurdlib.a}
and some binaries. All generated files are placed in a build directory inside
nurdlib named according to the compiler and build mode. The name of the build
directory is printed by the Makefile.

Early in the build process, an \emph{MD5} sum is calculated from source files
which have an impact on the online communication. The sum stored in the
library and the client must match at runtime to be sure that the data handling
in both processes are compatible.

Auto-configuration of flags and libraries is carried out with \emph{nconf}.
This step is performed early in the build, following the usual make dependency
rules.

To build a \emph{TRLO II}-enabled version, the \emph{TRLO II} library and the
appropriate firmware (http://fy.chalmers.se/~f96hajo/trloii/) must be fetched
and built, and the following environment variable must be set:
\begin{description}
  \item[\emph{TRLOII\_PATH}] Root directory of your \emph{trloii/} repository.
\end{description}
The master makefile will try to figure out the firmware provided by that TRLO
II installation, but it can also be set explicitly with the following
environment variables:
\begin{description}
  \item[\emph{TRIDI\_FW}] The TRIDI TRLO II firmware version.
  \item[\emph{VULOM4\_FW}] The VULOM4(b) TRLO II firmware version.
\end{description}

If you would like to build only a sub-set of all available module types, you
can specify them in this file. There is a list of modules near the top of the
main makefile which can be copied and modified.
\begin{Verbatim}[frame=single,numbers=left]
$ echo "MODULE_LIST=caen_v775" > local.mk
\end{Verbatim}



\subsection{nconf}

nconf scans through a file looking for \emph{if NCONF\_m<module>\_b<branch>},
groups all switches with the same \emph{module}, tries to build, link, and
execute a test program for every \emph{branch}, and whichever branch succeeds
first within a module is chosen.

To only use the results of nconf of another project, for example to use
nurdlib decisions in your own DAQ build:
\begin{verbatim}
NCONF\_ARGS = $(NURDLIB_PATH)/$(BUILD_DIR)/nconf.args
\end{verbatim}
If you access the four variables \emph{NCONF\_CPPFLAGS}, \emph{NCONF\_CFLAGS},
\emph{NCONF\_LDFLAGS}, \emph{NCONF\_LIBS}, you will get the flags that were
chosen during the nurdlib build process.

If you'd like to use nconf yourself, first:
\begin{verbatim}
NCONF\_H = all files to be nconfed
NCONF\_ARGS = $(BUILD_DIR)/nconf.args
NCONF\_PREV = $(NURDLIB_PATH)/$(BUILD_DIR)/nconf.args
include $(NURDLIB_PATH)/nconf/nconf.mk
\end{verbatim}
This will create files in \emph{\$(BUILD\_DIR)/nconf*}.

\emph{\$(BUILD\_DIR)/nconfing/} will contain intermediary files used when
nconf is running its tests.

\emph{\$(BUILD\_DIR)/nconf/} will contain the header, arguments, and log
files. The header defines the branch for each module and the arguments file
contains the accumulation of the previous nconf:ed arguments and those for the
current file.

\emph{\$(BUILD\_DIR)/nconf.args} will contain all the arguments from all
files.



\subsection{ntest}

\emph{nurdlib} has a rigorous set of unit tests that anybody can run to verify
a large fraction of the code. New features and critical bugs should spawn
testing code. This way, new code should not break existing tested code, and
old bugs should be detected to the extent of the tests, just the typical
testing approach. Note that this testing only applies to internal logic and
utilities, and can not test hardware or external user implementations!
\begin{Verbatim}[frame=single,numbers=left]
$ make test
\end{Verbatim}
The build system can also report the coverage from testing with \emph{gcov}.
Code built with coverage analysis runs very slowly so it is enabled only for
the build type \emph{cov}. The full routine to create coverage information
from tests is:
\begin{Verbatim}[frame=single,numbers=left]
$ make BUILD_TYPE=cov test      # Testing with coverage (much slow!).
$ make BUILD_TYPE=cov cov       # Terse summary.
$ make BUILD_TYPE=cov cov_files # Summary per file.
$ make BUILD_TYPE=cov cov_funcs # Summary per function.
$ make BUILD_TYPE=cov cov_anno  # Annotates files in $(BUILD_DIR)/cov/.
\end{Verbatim}




\section{Big features}



\subsection{Control interface}

While running, \emph{nurdlib} can provide information on crate and module
configuration and allow the user to talk to configured modules. This control
interface is served by a UDP server inside the library. Since it does eat some
resources it can be easily disabled (see the global \emph{control\_port}
config in \emph{cfg/default/global.cfg}).

In order to talk to this control server, use \emph{bin/nurdctrl}; this . The
help text should explain this program sufficiently well, but here are two
examples for fun and games:
\begin{Verbatim}[frame=single,numbers=left]
./nurdctrl -c
\end{Verbatim}
Lists configured crates in the nurdlib running on localhost on the default
port.
\begin{Verbatim}[frame=single,numbers=left]
./nurdctrl -a192.168.0.100:10000 -i0 -j 1 -d
\end{Verbatim}
This will dump the hardware registers in module 1 (zero-based indexing) in
crate 0, with \emph{nurdlib} running on the host 192.168.0.100 on port 10000.




\section{Configuration}

The config parser was implemented by hand to save memory for systems with
limited resources, so there's no handy list of grammar rules for the parsing.
This section aims to explain most of the grammar.

\begin{Verbatim}[frame=single,numbers=left]
top: item_list

item_list: item_list item

item: block OR config

block: block_header { item_list }

block_header: name OR name(block_param_list)

block_param_list: block_param_list block_param

block_param: scalar

config: name = value

value: scalar OR vector

vector: ( scalar_list )

scalar_list: scalar_list scalar

scalar: literal_string OR range OR value unit

range: integer .. integer

value: integer OR double

unit: mV OR uV etc...
\end{Verbatim}




\section{Miscellaneous}

This section aims to touch upon a few obscure points in the \emph{nurdlib}
build system and source code, for those interested in understanding all
constructs.



\subsection{Build system}

The build system is based on GNU make, and has been tested with at least
version 3.75. There is a single non-recursive makefile, which means there is
one huge dependency DAG. Invocation of make can therefore be slow because a
lot of dependencies are checked, but spurious cleans should be less prevalent.

Each sub-directory contains a makefile fragment named \emph{rules.mk}, and
each such file should define a unique name and the list of sources in a
name-referenced variable \emph{\$(SRC\_\$(NAME))}. A root-level makefile
snippet \emph{make.mk} is then included which provides common rules to build
output files for the given sources. At last, the master \emph{Makefile} binds
it all together.

Some modules need to access hardware via reading from and writing to certain
memory addresses. Rather than hard coding the addresses by eye and hand from
manuals into large lists of pointers, structs with appropriately offset
members are generated from ASCII based register list files which can be
created by copying and pasting. This also allows error checking and generating
introspection capable source code. This is provided by other makefile snippets
and register generating programs. Depending on the complexity of the module,
two such setups have been implemented (simply put, it's Struck SIS3316 vs the
rest).

Due to the strict compilation flags used in \emph{nurdlib}, sloppy system
headers can wreak havok during building. The selected solution was to make
local copies of such headers and modify the offending text. These replacements
can be found near the end of \emph{Makefile}.



\subsection{nconf}

Nurdlib uses a custom small auto-configuration tool called ''nconf''. Let's
start with an example:
\begin{Verbatim}[frame=single,numbers=left]
dir/bananas.h:
  #include <nconf/dir/myheader.h>
  #if NCONF_mBANANAS_bSTRAIGHT
  #  define NCONF_SRC bananas.c
  #  define NCONF_NOEXEC
  #  define BANANAS_BE_STRAIGHT 1
  #elif NCONF_mBANANAS_bBENT
  #  define NCONF_CPPFLAGS -I.
  #  define NCONF_CFLAGS -Wextra
  #  define NCONF_LDFLAGS -L.
  #  define NCONF_LIBS -lbent
  #  define NCONF_SRC bananas.c
  #  define NCONF_NOLINK
  #  define BANANAS_BE_BENT 1
  #elif NCONF_mBANANAS_bTWISTED
  #  define NCONF_SRC bananas.c
  #  define BANANAS_BE_TWISTED 1
  #endif
  #if NCONFING_mBANANAS
  #  define NCONF_TEST success == bananas_init()
  #  define NCONF_TEST_VOID bananas_volatile()
  #endif
\end{Verbatim}
''nconf'' parses the header and looks for module-branch macros which follow
the pattern ''NCONF\_m<module>\_b<branch>'', in this example
''BANANAS:STRAIGHT'' and ''BANANAS:BENT''. Every branch of every module is
tested until a branch compiles, links, and executes, after which no more
branches of that module is considered. Only one branch of each module is
active at any time, and while nconf:ing, only a single branch in the whole
file is active.

''NCONFING\_mBANANAS'' is active only while nconf:ing the ''BANANAS'' module.

''NCONF\_TEST'', if defined, is executed and the value is used for the exit
code of the test program. ''NCONF\_TEST\_VOID'', if defined, is executed. Note
that a void statement can be optimized away if it's not for example volatile!

The nconf:ing can be controlled per branch with extra macros:
\begin{Verbatim}[frame=single,numbers=left]
  NCONF\_CPPFLAGS -- added to CPPFLAGS
  NCONF\_CFLAGS   -- added to CFLAGS
  NCONF\_LDFLAGS  -- added to LDFLAGS
  NCONF\_LIBS     -- added to LIBS
  NCONF\_SRC      -- compiled and linked in one step with the main object
  NCONF\_NOEXEC   -- skips the execution step of the test-code
  NCONF\_NOLINK   -- skips the linking step
\end{Verbatim}

While nconf:ing, files are written to ''output\_dir/nconfing/nconf/filename*'',
for example ''mybuild/nconfing/nconf/dir/bananas.h''. When nconf finishes, it
writes the final files at ''mybuild/nconf/dir/bananas.h''.

The files produced during nconf:ing are:
\begin{Verbatim}[frame=single,numbers=left]
  filename     -- module-branch pair macro
  filename.c   -- test code
  filename.log -- log of compilation, linking, and execution
  filename.o   -- compiled test code
\end{Verbatim}

And the final files are:
\begin{Verbatim}[frame=single,numbers=left]
  filename      -- module-branch pair macro
  filename.args -- compilation and linking flags
\end{Verbatim}

The args file looks like:
\begin{Verbatim}[frame=single,numbers=left]
  CPPFLAGS
  CFLAGS
  LDFLAGS
  LIBS
\end{Verbatim}
Such a file can also be used as an input to ''nconf'', which will then be used
together with ''NCONF\_'' flags, and will also be included in the final args
file. This way flags can be accumulated among all nconf:ed files in a project.
Be vigilant! Certain flags override others (e.g. search path order matters,
and -D flag duplicates are removed lexicographically), and unfortunately it's
incredibly complicated to make the computer resolve such things!

So the following might happen in this example:
\begin{Verbatim}[frame=single,numbers=left]
  Command: nconf -cgcc-2.95 -ibananas.h -obuild -ainit.args
  nconf reads init.args and will use flags when compiling/linking
  nconf parses bananas.h
  nconf finds BANANAS:STRAIGHT
    Let's say base=build/nconfing/nconf/bananas.h
    NCONF_mBANANAS_bSTRAIGHT is defined to 1 in base
    Test code is written to base.c
    base.c is compiled with gcc-2.95 into base.o
    base.o is linked together with bananas.c, let's assume this fails
    Even if base.o linked, it would not have been executed because of NOEXEC
  nconf finds BANANAS:BENT
    NCONF_mBANANAS_bBENT is defined to 1 in base
    Test code is written to base.c
    base.c is compiled into base.o
    base.o is not linked because of NOLINK
    Success, the module BANANAS will not be tested any more
  nconf finds no new module
    Let's say base=build/nconf/bananas.h
    NCONF_mBANANAS_bBENT is defined to 1 in base
    The compilation and linking flags from init.args and the branch are written to base.args
\end{Verbatim}



\subsection{Coding style}

We try to follow the OpenBSD KNF style for all code, but there are for sure
always some exceptions...




\section{Features}



\subsection{Crate-level}


\subsubsection{Adaptive Conversion Time, ACVT}

ACVT is a potential readout optimization, due to the conversion time in data
acquiring modules. Polling modules is time-consuming, which together with low
resolution sleeping in the readout processor makes it undesirable. Some
systems solve this using a fixed Conversion Time, CVT, in hardware with a high
resolution sleep function. For optimal performance, the CVT should be as small
as possible while still.

It is possible in some instances to adjust the CVT. If a module can always
report data on any event, e.g. empty events even with data suppression,
\emph{nurdlib} can reduce the CVT while polling. If a poll fails, the CVT is
bumped up, and the process continues.

Note that this only works for modules which delive data on \emph{every} event,
no matter what. If you're experiencing data timeout errors, you will have to
disable ACVT and use a fixed value.


\subsubsection{Multi-event readout}

Multi-event readout is almost always a faster way of reading out a DAQ, and
should be used when possible. Nurdlib supports reading any amount of data from
its configured modules.


\subsubsection{Early dead-time release}

Modules in which conversion and slow readout from storage are decoupled can
make use of early dead-time release to reduce the per-event-pair dead-time.


\subsubsection{Shadow readout}

Shadow readout could also be called mutl-event live-time readout, where the
module conversion and buffer readout can execute in parallel. It's a slightly
more advanced version of early dead-time release readout.

\subsection{American versus British}

The source code has been written in American s,
whereas this document was written in British English. The main purpose to
point this out is to still the minds of observant and inquisitive readers.



\begin{thebibliography}{100}
\bibitem{bib:drasi} http://fy.chalmers.se/~f96hajo/drasi/doc/
\end{thebibliography}




\end{document}
